\documentclass[a4paper,12pt]{texMemo} 

\usepackage[spanish]{babel}
\selectlanguage{spanish}

\usepackage[utf8]{inputenc}
\renewcommand{\thesection}{\Roman{section}} 

\usepackage{parskip} % Adds spacing between paragraphs
\setlength{\parindent}{15pt} % Indent paragraphs

\usepackage{setspace}
\doublespacing
%\onehalfspacing

%	MEMO INFORMATION
\memoto{Godofredo Miguel Huerta Barrón\\ \emph{Director General de Seguridad Democrática, Ministerio del Interior.}} 
\memofrom{Dr. José Manuel Magallanes, PhD.\\ \emph{Consultor.}} 
\memosubject{Informe de Actividades realizadas en el marco de la consultoría para brindar el \emph{Servicio Especializado de Sistematización de Registros y Sistemas de Víctimas de Trata de Personas} .} 
\memoref{Orden de Servicio 4031 - Segundo Entregable: Análisis de los registros de las Bases de Datos existentes.} 
%\memodate{28 de Noviembre de 2017.}
\memodate{\today.}
\logo{\includegraphics[width=0.4\textwidth]{logo.png}} % Institution logo at the top right
%-----

\usepackage{Sweave}
\begin{document}
\Sconcordance{concordance:Informe2.tex:Informe2.Rnw:%
1 25 1 1 0 94 1}


\clearpage\maketitle
\thispagestyle{empty}
%----------------------------------------------------------------------------------------
%	MEMO CONTENT
%----------------------------------------------------------------------------------------

Tengo el agrado de dirigirme a usted, a fin de informarle, en esta segunda ocasión, sobre las acciones llevadas a cabo en el marco de la consultoría de la referencia. 

\section{Antecedentes}

A partir del primer informe que el suscrito produjo, fechado 17 de noviembre, quedó claro que este trabajo tiene por fin lograr integrar bases de datos que existen en diversas entidades del Estado, en particular, de instituciones tales como el Ministerio de la Mujer y Poblaciones Vulnerables (MIMP), el Ministerio de Justicia (MINJUS), el Ministerio Público (MP) y el propio Ministerio de Interior (MININTER), dónde la Policía Nacional del Perú (PNP) cuenta con sistemas que han colectado datos relevantes para los fines de este trabajo. Tal integración será un paso clave para apoyar a que el MININTER, y en particular a su Dirección General de Seguridad Democrática (DGSD), logre del Objetivo Estratégico \textnumero 7 (OE7): \emph{Elaborar la línea de base del Plan Nacional de Acción contra la Trata de Personas 2017-2021}.


\section{Acciones y/o coordinaciones llevadas a cabo}

A la fecha, se han llevado a cabo las siguientes actividades:

\begin{enumerate}
\item Con fecha 14 de noviembre de 2017, se recibió la Orden de Servicio 4031, donde se nos encarga el servicio motivo de la consultoría.

\item Del 15 al 17 de noviembre de 2017, se coordinó con el área usuaria o cliente del MININTER (DGSD) las especificaciones y expectativas concretas de este trabajo. Durante las coordinaciones, se contó con el apoyo directo del personal de la Dirección de Gestión del Conocimiento para la Seguridad (DGCS) del MININTER.

\item Del 17 al 18 de noviembre 2017, se procedió a elaborar el primer entregable, así como a planificar las acciones a seguir.

\item Del 18 al 21 de noviembre 2017, se ha venido coordinando de manera estrecha con el personal de la DGCS y la DGSD detalles específicos en cuanto a lo estrictamente necesario para el segundo informe. De ello quedo claro, que este informe debe aclarar qué bases de datos deben formar parte de este trabajo, explicar su necesidad, así como inventariar los contenidos de los registros de cada una de ellas para poder construir la base final ya integrada.

\item Del 21 al 24 de noviembre 2017, se ha coordinado estrechamente con la DGCS, a fin de recibir información, en forma de \emph{manuales}, sobre las bases de datos candidatas a formar parte de este trabajo.

\item Del 23 al 27 de noviembre 2017, se ha venido elaborando este segundo informe.

\end{enumerate}

\section{Insumos recibidos}

Se tiene en claro a partir de las coordinaciones con la DGCS y la GCSD cuales son las bases de datos necesarias y se ha enviado al consultor los manuales siguientes:
\begin{itemize}
\item El \emph{Manual de Usuario} del Sistema de Registro y Estadística del delito de Trata de Personas y afines (RETA) de la PNP, en particular del \emph{módulos de registro de denuncias}.
\item El \emph{Manual de Procedimiento} del Sistema de Información Estratégica sobre Trata de Personas (SISTRA) del 
Ministerio Público.
\item El \emph{Manual de Usuario} del  Sistema de Información Policial “ESINPOL” de la PNP.
\item El \emph{Manual de Usuario} del Modulo de Investigación Criminal del proyecto "adquisión de servidores de alta disponibilidad, equipos informáticos, aplicativos y migración de datos para la PNP".
\item el \emph{Manual de Usuario} del Sistema de Denuncias Policiales “SIDPOL”
\end{itemize}

No se ha hecho entrega al consultor de ningún manual de usuario de las bases de datos en posesión del MIMP. Sin embargo, si se ha discutido lo que se espera pueda ser compartido por algunas unidades del MIMP, como los \emph{Centros de Emergencia Mujer} (CEM), los \emph{Centros de Atención Residencial} (CAR), y los datos que pueda tener la \emph{Dirección de Investigación Tutelar}.

\section{Cumplimiento}

En el presente informe se adjunta el segundo entregable, dentro del plazo indicado en la orden de servicio de la referencia. Este entregable analiza la información proporcionada y además, conforme a la indicado por la DGCS y la DGSD, hemos producido un inventario de lo necesario a nivel de registros para producir la nueva base de datos integrada. El segund informe consta de XXXXXXXX páginas. 

\section{Condiciones para el siguiente entregable}

En el primer entregable se había considerado que la data que la DGSD quiere entregar tenía que haber estado disponible hasta el 24 de noviembre de 2017, para llevar a cabo el análisis de la misma. Sin embargo, luego de las coordinaciones con la DGCS y la GDSD, se aclaró que lo que se quería era analizar no la data en sí, si no la información disponible respecto a la data (metadata), según se detalle en los manuales que se enviaron al consultor. En ese sentido, venciendo la fecha para el entregable final al 29 del 12 de 2017, se espera que la data esté a más tardar el 1 de diciembre de 2017. De nos ser así, el trabajo sólo incluirá los datos que la DGCS y la GDSD envíen al consultor. Tales datos no son propiedad ni de la DGCS ni la GDSD, por lo que ellos tampoco pueden garantizar contar con todo los datos que están pidiendo.

\section{Solicitud de pago}

Habiendo cumplido con lo estipulado en la orden de servicio de la referencia, adjunto mi recibo por honorarios profesionales \textnumero E001- 31 por el 30\% del monto pactado (S/. 9600 de S/. 32000); a fin de que, con su aprobación a este segundo entregable, se pueda cancelar tal monto. Así mismo, se adjunta  con este informe el entregable en sí como documento aparte, el formato de declaración jurada entregado,  la carta de autorización de depósito en mi cuenta del BBVA Banco Continental vía código interbancario (CCI), y una copia de la orden de servicio de la referencia.

\section{Conclusiones}

De lo informado en este documento, se tiene en claro que:

\begin{itemize}

\item Es importante para la DGSD que el consultor le explique cómo se van a integrar los datos de tan diversas entidades del Estado.
\item La DGSD necesita saber qué información considera el consultor debe estar en la nueva base de datos integrada a partir de las bases de datos de interés.
\item La DGSD necesita saber cómo se integrará la información que se estará integrando en una nueva base de datos.
\item La DGSD necesita saber qué unidades de análisis se están considerando para la base de datos integrada, y en que otras unidades estás se podran integrar.
\item La DGSD necesita que se propongan indicadores que les permita tener una mirada global, y a su vez detectar casos que ameriten una mayor investigación posterior (tal investigación no es tarea del consultor).
\item El consultor podrá armar la base de datos integrados con los datos que la DGCS y la GDSD le entreguen hasta el 1 de diciembre, pues el trabajo se entrega el 29 de diciembre, conforme al plan de trabajo y la disponibilidad del consultor para este trabajo. La data debe pasar por varias etapas como se informó en el \emph{plan de trabajo} (primer entregable).
\item El consultor entiende que está cumpliendo con lo estipulado en la orden de servicio de la referencia y considera que se le debe abonar el primer pago por el 20\% del total.

\end{itemize}


Es todo cuanto se informa para su conocimiento y fines que se sirva determinar.




\vspace{75px}
\noindent
%\includegraphics[width=6cm]{firma}\\
{\bf Dr. José Manuel Magallanes, Ph.D.}\\
DNI: {\bf 07255056}\\
Correo Electrónico: {\bf josemanuel.magallanes@gmail.com}\\
Teléfono móvil: {\bf +51 985 125 105}


%----------------------------------------------------------------------------------------

\end{document}
